
\graficarPDF{graf_ej2}{Estimación de la trayectoria.}{fig:ej2}
	Al igual que en el trabajo práctico anterior, siendo todos los estados observables se obtiene una estimación muy buena de la trayectoria del vehículo como se expone en la Figura \ref{fig:ej2}. También se puede comprobar que el filtro de Kalman funciona correctamente al analizar las innovaciones (Figura \ref{fig:2covinn}). Se puede ver que la correlación de las mismas se asemeja al de un proceso blanco.
\graficarPDF{graf_ej2_covinn}{Innovaciones de las posiciones y velocidades en $x^e$ e $y^e$.}{fig:2covinn}
\graficarPDFa{0 0 8.0cm 8.0cm}{graf_ej2_pos}{Posición y error de la misma en función del tiempo.}{fig:2pos}
	\Juan{Checkear ésto}
	Examinando la posición en $x^e$ e $y^e$ se podría entender que la aproximación es buena. Sin embargo, analizando el error entre la trayectoria real y la estimada se ve que tiene la misma forma que la de los coeficientes de la matriz de rotación (Figura \ref{fig:2theta}). 
\graficarPDF{graf_ej2_theta}{Valores de los coeficientes de $C^e_b$ en el tiempo.}{fig:2theta}

	\Juan{No me parece necesario poner este gráfico}
\graficarPDF{graf_ej2_vel}{Velocidad ej2}{fig:2vel}
