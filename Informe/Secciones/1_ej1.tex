	\Juan{Preguntar a Emi si podemos usar un \emph{approach} similar al otro tp o si está muy mal}
	Se define la variable de estado $\vect{x}(t)$ asociada al sistema como:
		\begin{equation*}
			\vect{x}(t) = \begin{bmatrix} {p_x}(t) \\[0.3em] p_y(t) \\[0.3em] {v_x}(t) \\[0.3em] v_y(t) \\[0.3em] c_{11}(t) \\[0.3em] c_{12}(t) \\[0.3em] c_{21}(t) \\[0.3em] c_{22}(t) \end{bmatrix} \qquad%
			\dot{\vect{x}}(t) = \begin{bmatrix} {\dot{p}_x}(t) \\[0.3em] \dot{p}_y(t) \\[0.3em] {\dot{v}_x}(t) \\[0.3em] \dot{v}_y(t) \\[0.3em] \dot{c}_{11}(t) \\[0.3em] \dot{c}_{12}(t) \\[0.3em] \dot{c}_{21}(t) \\[0.3em] \dot{c}_{22}(t) \end{bmatrix}
		\end{equation*}

	Así el modelo resulta:
		\begin{equation*}
			\Sigma:
			\begin{cases}
				\dvect{x}(t) = A\: \vect{x}(t) + B \: \vect{\xi}(t) \\
				\vect{y}(t) = C\: \vect{x}(t) + \vect{\eta}(t)
			\end{cases}
		\end{equation*}
	donde $\vect{\xi}(t)$ es el ruido de proceso y $\vect{\eta}(t)$ el ruido de medición. La matriz $A$ contiene la información de la dinámica del sistema, $B$ es la matriz de entrada del ruido de proceso, y la matriz de salida es $C$, que conecta los estados con las mediciones de posición y velocidad. Las matrices quedan caracterizadas por la ecuación \eqref{eq:matrix_cont}.
		\begin{equation}
			\label{eq:matrix_cont}
			A(t) = \begin{pmatrix}0&0&I&0&0&0&0&0\\[0.3em]0&0&0&I&0&0&0&0\\[0.3em]0&0&0&0&a^b_x&a^b_y&0&0\\[0.3em]0&0&0&0&0&0&a^b_x&a^b_y&\\[0.3em]0&0&0&0&0&\omega^b&0&0\\[0.3em]0&0&0&0&-\omega^b&0&0&0\\[0.3em]0&0&0&0&0&0&0&\omega^b\\[0.3em]0&0&0&0&0&0&-\omega^b&0\end{pmatrix} \quad%
			B = \begin{pmatrix}0&0&0\\[0.3em]0&0&0\\[0.3em]0&c_{11}&c_{12}\\[0.3em]0&c_{21}&c_{22}\\[0.3em]1&0&0\\[0.3em]-1&0&0\\[0.3em]1&0&0\\[0.3em]-1&0&0\end{pmatrix} \quad%
			C = \begin{pmatrix}I&0&0\\[0.3em]0&I&0\end{pmatrix}
		\end{equation}
	
	Para poder tratar el sistema en forma digital, es necesario definir un período de muestreo $T$, que en este caso es igual al periodo de muestreo del giróscopo \SI{100}{\Hz}. Así se tiene que la exponencial matricial es:
		\begin{equation*}
			e^{A\,t} \approx I + A\,T + \frac{(A\,T)^2}{2} + \hdots
		\end{equation*}

	Truncando a segundo orden y reemplazando por $A$ se obtiene:
		\begin{equation*}
			A_d = \begin{pmatrix}	I&0&T&0&a^b_x\:\frac{T^2}{2}&a^b_y\:\frac{T^2}{2}&0&0\\[0.3em]%
						0&I&0&T&0&0&a^b_x\:\frac{T^2}{2}&a^b_y\:\frac{T^2}{2}\\[0.3em]%
						0&0&I&0&a^b_x\:T-\:a^b_y\:\omega^b\:\frac{T^2}{2}&a^b_y\: T+a^b_x\:\omega^b\:\frac{T^2}{2}&0&0\\[0.3em]%
						0&0&0&I&0&0&a^b_x\:T-\:a^b_y\:\omega^b\:\frac{T^2}{2}&a^b_y\: T+a^b_x\:\omega^b\:\frac{T^2}{2}\\[0.3em]%
						0&0&0&0&I-\frac{(\omega\:T)^2}{2}&\omega\:T&0&0\\[0.3em]%
						0&0&0&0&-\omega\:T&I-\frac{(\omega\:T)^2}{2}&0&0\\[0.3em]%
						0&0&0&0&0&0&I-\frac{(\omega\:T)^2}{2}&\omega\:T\\[0.3em]%
						0&0&0&0&0&0&-\omega\:T&I-\frac{(\omega\:T)^2}{2}
		\end{pmatrix}
		\end{equation*}


	Con dicho modelo y las mediciones de posición, velocidad, aceleración no inercial y velocidad angular no inercial, se aplica el filtro de Kalman.
	
	%De esta forma el modelo en tiempo discreto es el siguiente:
%	
%		\begin{equation*}
%			\Sigma_{d}:
%			\begin{cases}
%				\vect{x}_{n + 1} = A_{d}\: \vect{x}_{n} + B_{d} \: \vect{\xi}_{n} \\
%				\vect{y}_{n} = C_{d}\: \vect{x}_{n} + \vect{\eta}_{n}
%			\end{cases}
%		\end{equation*}
%		
%	donde ahora se obtienen nuevas matrices para el caso discreto. En dicho caso, a partir de las ecuaciones de la cinemática, se deduce que la matriz de la dinámica del sistema es:
%
%		\begin{equation*}
%			A_{d} = \begin{bmatrix} I & IT & I\frac{T^2}{2} \\[0.3em] 0 & I & IT \\[0.3em] 0 & 0 & I \end{bmatrix}
%		\end{equation*}
%
%	La matriz de salida $C_{d}$ dependerá de lo que se esté midiendo. Por ejemplo para el caso en que se sensen todas las variables será:
%	
%		\begin{equation*}
%			C_{d} = \begin{bmatrix} I & 0 & 0 \\[0.3em] 0 & I & 0 \\[0.3em] 0 & 0 & I \end{bmatrix}
%		\end{equation*}
%		
%		Para el caso en que alguna de las variables no se esté midiendo, será necesario eliminar alguna de las tres filas (de $2\times2$) de la matriz. En concreto se omite la primera, segunda o tercera fila, dependiendo si no se sensase posicion, velocidad o aceleración respectivamente. 
%		En este modelo, la matriz $B$ se toma como la matriz identidad.
%	En cuanto a los ruidos, se hace la suposición que se tratan de ruidos blancos con matrices de covarianza $Q_{d}$ para el ruido de proceso y $R_{d}$ para el ruido de medición. Cabe aclarar que el hecho de que se trate de ruidos blancos no implica que $Q_{d}$ y $R_{d}$ sean diagonales. Ambas matrices dan información de la correlación componente a componente de los vectores de ruido.
%	
