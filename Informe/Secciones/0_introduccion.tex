
	Se considera un vehículo que se desplaza definiendo una trayectoria tal que la posición en cada instante resulta $\vect{p}(t)$ y con una velocidad $\vect{v}(t)$, definidas en un plano de coordenadas inerciales $[x^e,y^e]$ de acuerdo a:
	\begin{equation*}
		\vect{p^e}(t) = \begin{bmatrix} p^e_x(t) \\[0.3em] p^e_y(t) \end{bmatrix} \qquad%
		\vect{v^e}(t) = \begin{bmatrix} v^e_x(t) \\[0.3em] v^e_y(t) \end{bmatrix}
	\end{equation*}

	\begin{equation}
		\begin{cases}
			\vect{\dot{p}^e}(t) = \vect{v^e}(t)\\
			\vect{\dot{v}^e}(t) = \vect{a^e}(t)\\
		\end{cases}
		\label{eq:din_inercial}
	\end{equation}

	Sin embargo, el dato de la aceleración viene dado en el sistema de coordenadas (terna que se denominará $b$) del vehículo que cambia su orientación a medida que se mueve. Por lo tanto, se define la variable $\theta(t)$ como se ve en la 
	\Juan{Generar e insertar figura de los ejes del avión}
	Figura \ref{fig:coordenadas_avion} que describe el ángulo de rotación del vehículo. El otro dato que se provee de la trayectoria es la velocidad angular $\omega(t) = \frac{\dd \theta}{\dd t} (t)$. En consecuencia, con todo lo enunciado previamente se plantean las ecuaciones de la dinámica del vehículo en \eqref{eq:din_veh}.

	\begin{equation}
		\begin{cases}
			\vect{\dot{p}^e}(t) = \vect{v^e}(t)\\[1.4em]
			\vect{\dot{v}^e}(t) = C^e_b \; \vect{a^b}(t)\\[0.6em]
			{C^e_b} = \dot{C}^e_b \; \begin{pmatrix} 0& -\omega(t)\\[0.3em] \omega(t)&0 \end{pmatrix}
		\end{cases}
		\label{eq:din_veh}
	\end{equation}

	donde $C^e_b = \begin{pmatrix} \cos(\theta(t)) & -\sen(\theta(t))\\[0.3em] \sen(\theta(t)) & \cos(\theta(t))\end{pmatrix}$




