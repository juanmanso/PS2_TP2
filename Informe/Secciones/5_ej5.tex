	
	Se define la descomposición de \emph{Cholesky} como la matriz $X^{c}_k$ tal que:
	\begin{equation*}
		X_k = X^{c^{H}}_k X^{c}_k
	\end{equation*}	

	A partir de ello, se propone para el filtro de Kalman definir: 
	\begin{equation*}
		M_0 = \begin{bmatrix} R^{c}_k& C_k\; P^c_{k/k-1}& 0 \\[0.3em] 0& A_k\; P^c_{k/k-1}& B_k\; Q^c_k \end{bmatrix}
	\end{equation*}
	de modo que al factorizar $M^H_0 = Q\,R$ se obtiene que $A=R^H R$, es decir que $R$ es la factorización de \emph{Cholesky} de $A$. Para optimizar los cálculos, se le agrega una fila a dicha matriz para incorporar las mediciones. Entonces:
	\begin{equation*}
	M =  \begin{bmatrix} R^{c}_k& C_k\; P^c_{k/k-1}& 0 \\[0.3em] 0& A_k\; P^c_{k/k-1}& B_k\; Q^c_k \\[0.3em] -y^H\,R^{c^{-1}}_k & \hat{x}^H\, P^{c^{-1}}_{k/k-1} & 0 \end{bmatrix}
	\end{equation*}

	Como la matriz $R$ será triangular inferior (propiedad de la factorización \emph{QR}) se tiene:
	\begin{equation*}
	R^H = \begin{bmatrix} X&0&0\\[0.3em]Y&Z&0\\[0.3em]W_1&W_2&W_3\end{bmatrix} \text{ donde } \begin{cases} P_{k+1/k} = Z\,Z^H\\ \hat{x}_{k+1/k}=Z\,W^H_2\\ \hat{g}_k = -X\,W^H_1\end{cases}
	\end{equation*}

	Por lo tanto, el algoritmo de Kalman consta en calcular la matriz $M$ en cada iteración, realizar la factorización \emph{QR} y a partir de las submatrices de $R$ obtener las estimaciones de los estados, innovaciones y autocorrelación.

	Al correr el algoritmo se obtiene la trayectoria de la Figura \ref{fig:ej5} \Juan{Insertar figuras}
